\section{PROCEDURE FOR PAPER SUBMISSION}

\subsection{Selecting a Template (Heading 2)}

First, confirm that you have the correct template for your paper size. This 
template has been tailored for output on the US-letter paper size. 
It may be used for A4 paper size if the paper size setting is suitably modified.

\subsection{Maintaining the Integrity of the Specifications}

The template is used to format your paper and style the text. All margins, 
column widths, line spaces, and text fonts are prescribed; please do not alter 
them. You may note peculiarities. For example, the head margin in this template 
measures proportionately more than is customary. This measurement and others 
are 
deliberate, using specifications that anticipate your paper as one part of the 
entire proceedings, and not as an independent document. Please do not revise 
any 
of the current designations

\section{MATH}

Before you begin to format your paper, first write and save the content as a 
separate text file. Keep your text and graphic files separate until after the 
text has been formatted and styled. Do not use hard tabs, and limit use of hard 
returns to only one return at the end of a paragraph. Do not add any kind of 
pagination anywhere in the paper. Do not number text heads-the template will do 
that for you.

Finally, complete content and organizational editing before formatting. Please 
take note of the following items when proofreading spelling and grammar:

\subsection{Abbreviations and Acronyms} Define abbreviations and acronyms the 
first time they are used in the text, even after they have been defined in the 
abstract. Abbreviations such as IEEE, SI, MKS, CGS, sc, dc, and rms do not have 
to be defined. Do not use abbreviations in the title or heads unless they are 
unavoidable.

\subsection{Units}

\begin{itemize}

\item Use either SI (MKS) or CGS as primary units. (SI units are encouraged.) 
English units may be used as secondary units (in parentheses). An exception 
would be the use of English units as identifiers in trade, such as Ò3.5-inch 
disk driveÓ.
\item Avoid combining SI and CGS units, such as current in amperes and magnetic 
field in oersteds. This often leads to confusion because equations do not 
balance dimensionally. If you must use mixed units, clearly state the units for 
each quantity that you use in an equation.
\item Do not mix complete spellings and abbreviations of units: ÒWb/m2Ó or 
Òwebers per square meterÓ, not Òwebers/m2Ó.  Spell out units when they appear 
in 
text: Ò. . . a few henriesÓ, not Ò. . . a few HÓ.
\item Use a zero before decimal points: Ò0.25Ó, not Ò.25Ó. Use Òcm3Ó, not ÒccÓ. 
(bullet list)

\end{itemize}


\subsection{Equations}

The equations are an exception to the prescribed specifications of this 
template. You will need to determine whether or not your equation should be 
typed using either the Times New Roman or the Symbol font (please no other 
font). To create multileveled equations, it may be necessary to treat the 
equation as a graphic and insert it into the text after your paper is styled. 
Number equations consecutively. Equation numbers, within parentheses, are to 
position flush right, as in (1), using a right tab stop. To make your equations 
more compact, you may use the solidus ( / ), the exp function, or appropriate 
exponents. Italicize Roman symbols for quantities and variables, but not Greek 
symbols. Use a long dash rather than a hyphen for a minus sign. Punctuate 
equations with commas or periods when they are part of a sentence, as in

$$
\alpha + \beta = \chi \eqno{(1)}
$$

Note that the equation is centered using a center tab stop. Be sure that the 
symbols in your equation have been defined before or immediately following the 
equation. Use Ò(1)Ó, not ÒEq. (1)Ó or Òequation (1)Ó, except at the beginning 
of 
a sentence: ÒEquation (1) is . . .Ó

\subsection{Some Common Mistakes}
\begin{itemize}


\item The word ÒdataÓ is plural, not singular.
\item The subscript for the permeability of vacuum ?0, and other common 
scientific constants, is zero with subscript formatting, not a lowercase letter 
ÒoÓ.
\item In American English, commas, semi-/colons, periods, question and 
exclamation marks are located within quotation marks only when a complete 
thought or name is cited, such as a title or full quotation. When quotation 
marks are used, instead of a bold or italic typeface, to highlight a word or 
phrase, punctuation should appear outside of the quotation marks. A 
parenthetical phrase or statement at the end of a sentence is punctuated 
outside 
of the closing parenthesis (like this). (A parenthetical sentence is punctuated 
within the parentheses.)
\item A graph within a graph is an ÒinsetÓ, not an ÒinsertÓ. The word 
alternatively is preferred to the word ÒalternatelyÓ (unless you really mean 
something that alternates).
\item Do not use the word ÒessentiallyÓ to mean ÒapproximatelyÓ or 
ÒeffectivelyÓ.
\item In your paper title, if the words Òthat usesÓ can accurately replace the 
word ÒusingÓ, capitalize the ÒuÓ; if not, keep using lower-cased.
\item Be aware of the different meanings of the homophones ÒaffectÓ and 
ÒeffectÓ, ÒcomplementÓ and ÒcomplimentÓ, ÒdiscreetÓ and ÒdiscreteÓ, ÒprincipalÓ 
and ÒprincipleÓ.
\item Do not confuse ÒimplyÓ and ÒinferÓ.
\item The prefix ÒnonÓ is not a word; it should be joined to the word it 
modifies, usually without a hyphen.
\item There is no period after the ÒetÓ in the Latin abbreviation Òet al.Ó.
\item The abbreviation Òi.e.Ó means Òthat isÓ, and the abbreviation Òe.g.Ó 
means 
Òfor exampleÓ.

\end{itemize}
